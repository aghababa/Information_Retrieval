\documentclass[11pt]{article}

\usepackage{pgfplots}
\pgfplotsset{compat=1.12}
\newtheorem{define}{Definition}
%\newcommand{\Z}{{\mathbb{Z}}}
%\usepackage{psfig}
\usepackage{tikz}
\usetikzlibrary{arrows, automata}
\oddsidemargin=0.15in
\evensidemargin=0.15in
\topmargin=-.5in
\textheight=9in
\textwidth=6.25in
\usepackage{latexsym,bm}
\usepackage{amsmath}
\usepackage{amsfonts}%for math
\usepackage{graphicx}%for eps
\usepackage{url}


\newif\ifgrading

\gradingtrue
%\gradingfalse

\begin{document}
\input{preamble.tex}

\assignment{2}{}{u1255635}{Hasan Pourmahmoodaghababa}

\baselineskip = 5mm

%\documentclass[11pt]{article}

\usepackage{pgfplots}
\pgfplotsset{compat=1.12}
\newtheorem{define}{Definition}
%\newcommand{\Z}{{\mathbb{Z}}}
%\usepackage{psfig}
\usepackage{tikz}
\usetikzlibrary{arrows, automata}
\oddsidemargin=0.15in
\evensidemargin=0.15in
\topmargin=-.5in
\textheight=9in
\textwidth=6.25in
\usepackage{latexsym,bm}
\usepackage{amsmath}
\usepackage{amsfonts}%for math
\usepackage{graphicx}%for eps
\usepackage{url}


\newif\ifgrading

\gradingtrue
%\gradingfalse

\begin{document}
\input{preamble.tex}

\assignment{3}{}{u1255635}{Hasan Pourmahmoodaghababa}

\baselineskip = 5mm

%\documentclass[11pt]{article}

\usepackage{pgfplots}
\pgfplotsset{compat=1.12}
\newtheorem{define}{Definition}
%\newcommand{\Z}{{\mathbb{Z}}}
%\usepackage{psfig}
\usepackage{tikz}
\usetikzlibrary{arrows, automata}
\oddsidemargin=0.15in
\evensidemargin=0.15in
\topmargin=-.5in
\textheight=9in
\textwidth=6.25in
\usepackage{latexsym,bm}
\usepackage{amsmath}
\usepackage{amsfonts}%for math
\usepackage{graphicx}%for eps
\usepackage{url}


\newif\ifgrading

\gradingtrue
%\gradingfalse

\begin{document}
\input{preamble.tex}

\assignment{3}{}{u1255635}{Hasan Pourmahmoodaghababa}

\baselineskip = 5mm

%\documentclass[11pt]{article}

\usepackage{pgfplots}
\pgfplotsset{compat=1.12}
\newtheorem{define}{Definition}
%\newcommand{\Z}{{\mathbb{Z}}}
%\usepackage{psfig}
\usepackage{tikz}
\usetikzlibrary{arrows, automata}
\oddsidemargin=0.15in
\evensidemargin=0.15in
\topmargin=-.5in
\textheight=9in
\textwidth=6.25in
\usepackage{latexsym,bm}
\usepackage{amsmath}
\usepackage{amsfonts}%for math
\usepackage{graphicx}%for eps
\usepackage{url}


\newif\ifgrading

\gradingtrue
%\gradingfalse

\begin{document}
\input{preamble.tex}

\assignment{3}{}{u1255635}{Hasan Pourmahmoodaghababa}

\baselineskip = 5mm

%\input{answers}

\section*{Ranking Features}%

\begin{itemize}
\item Query-document matching features: sum of tf$*$idf, mean of term frequency, BM25.
\item Document quality features: QualityScore, QualityScore2.
\item User behavior features: url dwell time, url click count, Query-url click count.
\end{itemize}
About which type of features are expected to be most useful in practice, I think it depends on many things like the query and the intend of user. 
\begin{itemize}
\item There are cases that query-document matching features are expected to be most useful in practice (like when we are searching for the query "Facebook"). 
\item There are cases that document quality features are very important (for example, when one searches for an app to install on his/her machine. In this case the quality of a web becomes essential).
\item In other situations like searching for an ambiguous query (probably like "Java") with more than one meaning, in addition to query-document matching features, user behavior features also can be effective to help semantic matching. 
\end{itemize}

%It seems that Query-document matching features are expected to be most useful in practice as more than $90\%$ of features are of this type which shows their importance. 



\section*{Ranking Models}%

\begin{itemize}
\item SVMrank: Pairwise
\item RankNet: Pairwise
\item MART: Pointwise
\item LamdbaMART: Listwise
\end{itemize}
For nDCG@5 I got the following values:
\begin{itemize}
\item SVMrank: 0.2975 (using linear normalization and sorted data by qid. I wrote a code in python to calculate nDCG@5 from predictions generated by SVMrank for test data.)
\item RankNet: 0.2893 (using zscore normalization, 1 layer, 5 nodes and learning rate of 0.00001)
\item MART: 0.4294 (using linear normalization)
\item LamdbaMART: 0.4311 (using linear normalization)
\end{itemize}
LamdbaMART outperformed other models here. I think the reason is that it is a listwise method and thus is based on ranking metrics (like MAP, nDCG) on list level. Therefore, to maximize ranking performance, this model tries to optimize the ranking list directly. However, in pointwise and pairwise models the loss functions we try to optimize are not the ranking metrics. Therefore, it is expected to get a better performance with listwise models like LamdbaMART than pointwise and pairwise models like SVMrank, RankNet and MART. 












\end{document}


\section*{Ranking Features}%

\begin{itemize}
\item Query-document matching features: sum of tf$*$idf, mean of term frequency, BM25.
\item Document quality features: QualityScore, QualityScore2.
\item User behavior features: url dwell time, url click count, Query-url click count.
\end{itemize}
About which type of features are expected to be most useful in practice, I think it depends on many things like the query and the intend of user. 
\begin{itemize}
\item There are cases that query-document matching features are expected to be most useful in practice (like when we are searching for the query "Facebook"). 
\item There are cases that document quality features are very important (for example, when one searches for an app to install on his/her machine. In this case the quality of a web becomes essential).
\item In other situations like searching for an ambiguous query (probably like "Java") with more than one meaning, in addition to query-document matching features, user behavior features also can be effective to help semantic matching. 
\end{itemize}

%It seems that Query-document matching features are expected to be most useful in practice as more than $90\%$ of features are of this type which shows their importance. 



\section*{Ranking Models}%

\begin{itemize}
\item SVMrank: Pairwise
\item RankNet: Pairwise
\item MART: Pointwise
\item LamdbaMART: Listwise
\end{itemize}
For nDCG@5 I got the following values:
\begin{itemize}
\item SVMrank: 0.2975 (using linear normalization and sorted data by qid. I wrote a code in python to calculate nDCG@5 from predictions generated by SVMrank for test data.)
\item RankNet: 0.2893 (using zscore normalization, 1 layer, 5 nodes and learning rate of 0.00001)
\item MART: 0.4294 (using linear normalization)
\item LamdbaMART: 0.4311 (using linear normalization)
\end{itemize}
LamdbaMART outperformed other models here. I think the reason is that it is a listwise method and thus is based on ranking metrics (like MAP, nDCG) on list level. Therefore, to maximize ranking performance, this model tries to optimize the ranking list directly. However, in pointwise and pairwise models the loss functions we try to optimize are not the ranking metrics. Therefore, it is expected to get a better performance with listwise models like LamdbaMART than pointwise and pairwise models like SVMrank, RankNet and MART. 












\end{document}


\section*{Ranking Features}%

\begin{itemize}
\item Query-document matching features: sum of tf$*$idf, mean of term frequency, BM25.
\item Document quality features: QualityScore, QualityScore2.
\item User behavior features: url dwell time, url click count, Query-url click count.
\end{itemize}
About which type of features are expected to be most useful in practice, I think it depends on many things like the query and the intend of user. 
\begin{itemize}
\item There are cases that query-document matching features are expected to be most useful in practice (like when we are searching for the query "Facebook"). 
\item There are cases that document quality features are very important (for example, when one searches for an app to install on his/her machine. In this case the quality of a web becomes essential).
\item In other situations like searching for an ambiguous query (probably like "Java") with more than one meaning, in addition to query-document matching features, user behavior features also can be effective to help semantic matching. 
\end{itemize}

%It seems that Query-document matching features are expected to be most useful in practice as more than $90\%$ of features are of this type which shows their importance. 



\section*{Ranking Models}%

\begin{itemize}
\item SVMrank: Pairwise
\item RankNet: Pairwise
\item MART: Pointwise
\item LamdbaMART: Listwise
\end{itemize}
For nDCG@5 I got the following values:
\begin{itemize}
\item SVMrank: 0.2975 (using linear normalization and sorted data by qid. I wrote a code in python to calculate nDCG@5 from predictions generated by SVMrank for test data.)
\item RankNet: 0.2893 (using zscore normalization, 1 layer, 5 nodes and learning rate of 0.00001)
\item MART: 0.4294 (using linear normalization)
\item LamdbaMART: 0.4311 (using linear normalization)
\end{itemize}
LamdbaMART outperformed other models here. I think the reason is that it is a listwise method and thus is based on ranking metrics (like MAP, nDCG) on list level. Therefore, to maximize ranking performance, this model tries to optimize the ranking list directly. However, in pointwise and pairwise models the loss functions we try to optimize are not the ranking metrics. Therefore, it is expected to get a better performance with listwise models like LamdbaMART than pointwise and pairwise models like SVMrank, RankNet and MART. 












\end{document}


\section*{Question 1.}%
Language modeling approaches are text dependent but classical probabilistic models are not. Language modeling approaches give a probability distribution over word sequences and so a different approach for scoring documents according to a query. It helps us to measure the uncertainties associated with the use of natural language. In fact, using a language model one can sample word sequences according to that model to get a piece of text. Therefore, language models are appropriate for query generation. However, classical probabilistic models are utilized for document generation. Both models are useful for quantifying how likely a document is similar/relevant to a query but from different perspectives. All in all, they both are trying to score documents according to a query but from different approaches (language modeling are data-dependent but classical probabilistic models are not). 


\section*{Question 2.}
(a) The unigram language model probability of a term in a document is $L(d) = \Pi_{t \in d} \, p(t | \theta_d)$, where $d$ shows the document, $t$ shows a term, $\theta_d$ represents the distribution of $d$ and $L(d)$ is the likelihood of the document $d$. Notice terms are assumed to be independent of each other given a document. 

Now we are going to maximize the likelihood of the document using maximum likelihood estimation in order to obtain $p(t | \theta_d)$. To accomplish this we first take $\log$ from $L(d)$, which we denote by $LL(d)$, and try to maximize it (remember that $\log$ is an increasing function and so maximizing $LL(d)$ will result in maximizing $L(d)$). So we have $LL(d) = \sum_{t \in d} \log(p(t| \theta_d))$ as the objective function and $\sum_{t \in V} p(t| \theta_d) =1$ as the only constraint, where $V$ denotes the unique terms in $d$.  Applying Lagrangian multiplier method we can consider the function 
$$f(d, \lambda) = \sum_{t \in d} \log(p(t| \theta_d)) + \lambda (1 - \sum_{t \in V} p(t| \theta_d)),$$ 
as the objective function.  Taking the derivative of $f(d, \lambda)$ with respect to a term $p(t_1 | \theta_d)$ for some term $t_1$ we get 
$$\dfrac{\partial f(d, \lambda)}{p(t_1 | \theta_d)} = \sum_{t \in d, \, t=t_1} \frac{1}{p(t | \theta_d)} - \lambda = \frac{count(t_1, d)}{p(t_1 | \theta_d)} - \lambda.$$ 
Setting $\dfrac{\partial f(d, \lambda)}{\partial p(t_1 | \theta_d)} =0$ implies that $p(t_1 | \theta_d) = \frac{count(t_1, d)}{\lambda}$. 

Since this is true for any $t_1 \in d$, we use $t$ instead of $t_1$ thereafter. Now employing the condition $\sum_{t \in V} p(t| \theta_d) =1$ we get $\sum_{t \in V} \frac{count(t, d)}{\lambda} = 1$, which implies $\lambda = |d|$. Therefore, $p(t| \theta_d) = \frac{count(t, d)}{|d|}$ and the model is $MLE(d) =  \Pi_{t \in d} \, \frac{count(t, d)}{|d|}$.


(b) Probabilities are as follows:
\begin{itemize}
\item $p(Information) = \dfrac{count(Information, d)}{|d|} = \dfrac{4}{23} = 0.1739$,
\item $p(resources) = \dfrac{count(resources, d)}{|d|} = \dfrac{2}{23} = 0.0870$,
\item $p(system) = \dfrac{count(system, d)}{|d|} = \dfrac{1}{23} = 0.0435$.
\end{itemize}


\section*{Question 3.}%
Solving the zero probability problem is one of the main reasons for smoothing. In fact, if a term in a query does not occur in a document, that term would receive 0 probability and so the whole product in the model would be 0. This happens because using MLE we allow the model to overfit the data. To solve this problem we need to have $p(t| \theta_d) > 0$ for all terms $t$ in the vocabulary/corpus (not document). This will make sure that all documents have a chance, although a tiny chance, to be considered for a query and participate in scoring. 

The next reason I would say is that without smoothing, general terms in the query, like stop words, would get quite a high probability and so the score for a document will be affected by them. Smoothing overcomes this issue and helps to reduce the impact of general terms in scoring documents for a query. 

Lastly, smoothing methods are simple enough to use. 


\section*{Question 4.}%
We know that log-likelihood without JM smoothing uses $p_{MLE}(t| \theta_d) = \frac{count(t,d)}{|d|}$ and so it is 
$$\log p_{MLE}(q| \theta_d) = \sum_{t \in q} \log p_{MLE}(t| \theta_d) = \sum_{t \in q} \log \Big(\frac{count(t,d)}{|d|} \Big) = \log(\frac{4}{23}) + \log(\frac{1}{23}) + \log(\frac{0}{23}) = - \infty,$$
which, I guess, it means that this document is not considered relevant for that query as score is $-\infty$. Also, the likelihood is $e^{- \infty}= 0$.

However, the log-likelihood with JM smoothing utilizes 
$$p_{JM}(t| \theta_d) = \lambda \, p_{MLE}(t| d) + (1- \lambda) \, p_{MLE}(t | C) = \lambda \, \frac{count(t, d)}{|d|} + (1 - \lambda) \, \frac{count(t, C)}{|C|},$$ 
(here $\lambda = 0.5$) and thus log-likelihood with JM smoothing is 
$$\begin{array}{ll} \vspace{2mm}
\log p_{JM}(q| \theta_d) & = \sum_{t \in q} \log p_{JM}(t| \theta_d) = \sum_{t \in q}   \log(\frac{1}{2}) + \log\big(\frac{count(t, d)}{|d|} + \frac{count(t, C)}{|C|}\big) \\ \vspace{1mm} &
=3 \log(\frac{1}{2}) + \log(\frac{4}{23} + \frac{100}{200*200}) + \log(\frac{1}{23} + \frac{40}{200*200}) + \log(\frac{0}{23} + \frac{60}{200*200}) \\ \vspace{1mm} & 
= -13.4294.
\end{array}$$
So the likelihood is $e^{-13.4294}= 0.00000147$.

From this example we can learn that without smoothing if a query has at least one term that is not in a document, the log-likelihood will assign $-\infty$ score to that document and so it will not be retrieved as a relevant document to that query. Nevertheless, with smoothing, the model partially considers that term (assuming it is in the corpus) and so assigns a positive probability for that term. So it never will pass 0 to the $\log$ function and as a result no document will get a $-\infty$ as a score. Therefore, all the documents will get a chance to be retrieved even though they do not consist some terms from a query. 



\section*{Question 5.}%
Dirichlet prior smoothing is expected to perform best and additive smoothing is expected to perform the worst. Of course, JM smoothing also will perform well assuming the hyperparameter $\lambda$ is chosen appropriately. The reason that we expect additive smoothing would not perform well is that it is not incorporating the corpus information (like term count in the corpus) and the impact of document length is not appropriate. However, Dirichlet prior and JM smoothing consider the corpus effect but Dirichlet prior smoothing also take the length of the document appropriately into account (the $\lambda$ value is fixed for JM smoothing but it differs document by document for Dirichlet smoothing). So, we expect Dirichlet smoothing performs better than JM smoothing. 



\section*{Question 6.}%

{\bf First part:}
The order  for $\alpha \text{-iDCG}$ that I have chosen is $d_8, d_3, d_5, d_1, d_6, d_2, d_4, d_7$.

$\alpha \text{-iDCG} = 4+\frac{2.5}{\log_2 3}+\frac{1}{\log_2 4}+\frac{0.5}{\log_2 5}+\frac{0.25}{\log_2 6} = 6.3894$
\begin{itemize}
\item List 1: $\alpha\text{-DCG} = 1+\frac{2}{\log_2 3}+\frac{3}{\log_2 4}+\frac{0.5}{\log_2 7}+\frac{1.75}{\log_2 9} = 4.492 \Longrightarrow$ 

$\alpha\text{-nDCG} = \frac{\alpha\text{-DCG}}{\alpha\text{-iDCG}} = \frac{4.492}{6.3894}= {\bf 0.703}$

\item List 2: $\alpha\text{-DCG} = 1+\frac{2}{\log_2 3}+\frac{2.5}{\log_2 4}+\frac{0.25}{\log_2 7}+\frac{2.5}{\log_2 9} = 4.3896 \Longrightarrow$ 

$\alpha\text{-nDCG} = \frac{\alpha\text{-DCG}}{\alpha\text{-iDCG}} = \frac{4.3896}{6.3894}= {\bf 0.687}$
\end{itemize}

\hspace{-6mm}{\bf Second part:}

$\text{iDCG} = 4+\frac{3}{\log_2 3}+\frac{2}{\log_2 4}+\frac{1}{\log_2 5}+\frac{1}{\log_2 6} = 7.7103$
\begin{itemize}
\item List 1: $\text{DCG} = 1+\frac{2}{\log_2 3}+\frac{3}{\log_2 4}+\frac{1}{\log_2 7}+\frac{4}{\log_2 9} = 5.3799  \Longrightarrow$ 

$\text{nDCG} = \frac{\text{DCG}}{\text{iDCG}} = \frac{5.3799}{7.7103}= {\bf 0.6978}$

\item List 2: $\text{DCG} = 1+\frac{2}{\log_2 3}+\frac{4}{\log_2 4}+\frac{1}{\log_2 7}+\frac{3}{\log_2 9} = 5.5645 \Longrightarrow$ 

$\text{nDCG} = \frac{\text{DCG}}{\text{iDCG}} = \frac{5.5645}{7.7103}= {\bf 0.7217}$
\end{itemize}

\hspace{-6mm}{\bf Third part:} 

We can see that $\alpha\text{-nDCG}$ from List 1 to List 2 is decreased but $\text{nDCG}$ is increased. The main reason is that when we use $\alpha\text{-nDCG}$, we penalize similar items but in $\text{nDCG}$ we don't. In fact, the relevance value for each document (i.e. $r_i$) does not depend on its position in the list in $\text{nDCG}$-calculation but it (i.e. $rel(i)$) differs for $\alpha\text{-nDCG}$-calculation according to the novelty in covering subtopics. 

We know that the place of documents $d_3$ and $d_8$ is changed in Lists 1 and 2 and that $d_8$ is related to more subtopics than $d_3$ (i.e. $d_8$ has a higher relevance value than $d_3$ in $\text{nDCG}$-calculation). Therefore, $\text{nDCG}$ will be higher if $d_8$ appears before $d_3$. However, for $\alpha\text{-nDCG}$, document $d_8$ that contains more similar items than $d_3$ with respect to $d_1$ and $d_5$ (that have already been retrieved), will gain lower weight in $\alpha\text{-nDCG}$-calculation if $d_8$ appears before $d_3$. 

\section*{Question 7.}%

One reason is that when we force diversity, we are biasing the retrieval results towards  covering more diverse topics related to the query. It means that the model tries to prevent retrieving similar documents. So, without considering diversification, if a query is ambiguous, then it is more likely that the model retrieves completely non-relevant documents. On the other hand, by diversification, it is more probable that some of the retrieved results be related to the intent of the user. Thus, the relevance will be improved for such queries. 
For non-ambiguous queries, of course, the search result will focus on the right topic/keywords and retrieve the relevant documents, as the query is not ambiguous. Therefore, the over all relevance will be improved. 

% For example, in web search, it prevents near-duplicate webs to appear in the retrieval list. 























\end{document}
