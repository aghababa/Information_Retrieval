\documentclass[11pt,a4paper,reqno]{amsart}
\usepackage{amsmath, amsthm, amscd, amsfonts, amsthm, amssymb, graphicx}
\usepackage[bookmarksnumbered, colorlinks, plainpages]{hyperref}
\textwidth=15.5cm \textheight=22.000cm \topmargin=0.00cm
\oddsidemargin=0.00cm \evensidemargin=0.00cm \headheight=14.4pt
\headsep=1.2500cm \numberwithin{equation}{section}
\hyphenation{semi-stable} \emergencystretch=12pt
\usepackage[]{todonotes}


\def\thefootnote{*}
\newtheorem{theorem}{Theorem}[section]
\newtheorem{lem}[theorem]{Lemma}
\newtheorem{rem}{Remark}[section]
\theoremstyle{definition}
\newtheorem{definition}[theorem]{Definition}
\theoremstyle{notation}
\newtheorem{question}[theorem]{Question}


\def\Hom{\mathop{\rm Hom}}
\def\v{\mathop{\varphi}}


\renewcommand{\baselinestretch}{1.27}


%%% ----------------------------------------------------------------------

%%% ----------------------------------------------------------------------


\begin{document}


\title[Assignment 1 Solution (Information Retrieval)]
{Assignment 1 Solution (Information Retrieval)}


\author[Hasan Pourmahmoodaghababa]{Hasan Pourmahmoodaghababa \\ UID: u1255635 \\ h.pourmahmoodaghababa@utah.edu}
    \address{School of Computing, University of Utah, Utah, USA}
\email{h.pourmahmoodaghababa@utah.edu}


\maketitle

\baselineskip = 8mm

%%%%%%%%%%%%%%%%%%%%%%%%%%%%%%%%%%%%%%%%%%

\section{Evaluation Metrics}

\begin{question}%
\end{question}
\begin{itemize}
\item List 1: $\frac{1}{3}(1+1+1+0)=\frac{3}{3} = 1$
\item List 2: $\frac{1}{3}(1+0+0+0)=\frac{1}{3} = 0.3333$
\item List 3: $\frac{1}{3}(1+0+\frac{2}{3}+0)=\frac{5}{9}=0.5556$
\item List 4: $\frac{1}{3}(1+0+\frac{2}{3}+\frac{3}{4})=\frac{29}{36} = 0.8056$
\item List 5: $\frac{1}{3}(0+\frac{1}{2}+\frac{2}{3}+0)=\frac{7}{18} = 0.3889$
\end{itemize}


\begin{question}%
\end{question}

$iDCG = 4+\frac{3}{\log_2 3}+\frac{1}{\log_2 4} = 6.3928$
\begin{itemize}
\item List 1: $DCG = 4+\frac{3}{\log_2 3}+\frac{1}{\log_2 4}+\frac{0}{\log_2 5} = 6.3928 \Longrightarrow nDCG = \frac{DCG}{iDCG} = \frac{6.3928}{6.3928}=1$
\item List 2: $DCG = 3+\frac{0}{\log_2 3}+\frac{0}{\log_2 4}+\frac{0}{\log_2 5} = 3 \Longrightarrow nDCG = \frac{DCG}{iDCG} = \frac{3}{6.3928}= 0.4693$
\item List 3: $DCG = 1+\frac{0}{\log_2 3}+\frac{3}{\log_2 4}+\frac{0}{\log_2 5} = 2.5 \Longrightarrow nDCG = \frac{DCG}{iDCG} = \frac{2.5}{6.3928}= 0.3911$
\item List 4: $DCG = 4+\frac{0}{\log_2 3}+\frac{3}{\log_2 4}+\frac{1}{\log_2 5} = 5.9307 \Longrightarrow nDCG = \frac{DCG}{iDCG} = \frac{5.9307}{6.3928}= 0.9277$
\item List 5: $DCG = 0+\frac{4}{\log_2 3}+\frac{3}{\log_2 4}+\frac{0}{\log_2 5} = 4.0237 \Longrightarrow nDCG = \frac{DCG}{iDCG} = \frac{4.0237}{6.3928}= 0.6294$
\end{itemize}


%%%%%%%%%%%%%%%%%%%%%%%%%%%%%%%%%%%%%%%%%%

\section{Text Processing and Indexing}


\begin{question}%
\end{question}
\begin{itemize}
\item tokenization: We split the sentence by space and remove symbols like comma, dot, semicolon.
\begin{table}[h]
\begin{tabular}{|c|c|c|c|c|c|c|c|}
\hline
According & to & Wikipedia & information & technology & is & the & use 
\\ \hline 
of & computers & to & create & process & store & and & exchange
\\ \hline 
all & kinds & of & electronic & data & and & information &
\\ \hline
\end{tabular}
\end{table}



\item normalization: We change all words to the same written shape like changing them to lowercases.  
\begin{table}[h]
\begin{tabular}{|c|c|c|c|c|c|c|c|}
\hline
according & to & wikipedia & information & technology & is & the & use 
\\ \hline 
of & computers & to & create & process & store & and & exchange
\\ \hline 
all & kinds & of & electronic & data & and & information &
\\ \hline
\end{tabular}
\end{table}



\item stopping: We remove stop words like ``a, an, the, and, of, to, is, ...''. 
\begin{table}[h]
\begin{tabular}{|c|c|c|c|c|c|c|c|}
\hline
according & -- & wikipedia & information & technology & -- & -- & use  \\ \hline
-- & computers & -- & create & process & store & -- & exchange \\ \hline
 all & kinds & -- & electronic & data & -- & information & \\ \hline
\end{tabular}
\end{table}



\item Krovetz stemming: We change all words to their ``algorithmic + dictionary-based'' versions, like  plural to singular and normalizing verb tense.
\begin{table}[h]
\begin{tabular}{|c|c|c|c|c|c|c|c|}
\hline
accord & -- & wikipedia & inform & technology & -- & -- & use  \\ \hline
-- & computer & -- & create & process & store & -- & exchange \\ \hline
 all & kind & -- & electronic & data & -- & inform & \\ \hline
\end{tabular}
\end{table}
\end{itemize}
I treated the words $\{\text{the, and, of, to, is}\}$ as stopping as they don't cary a useful information and are frequent. 




\begin{question}%
\end{question}
\begin{itemize}
\item Inverted index is efficient for big cuprous' queries where reading each document sequentially can be inefficient. 
\item It depends on data. In fact if data we are exploring is not a big data, I assume there would be no considerable improvement; however, it would have a significant improvement in time efficiency for big data sets. For example, for a query to search within all webpages (like search in Google) inverted index has a dramatic improvement in the search system. 
\end{itemize}




\begin{question}%
\end{question}
\begin{itemize}
\item 
\begin{itemize}
\item $x_d = \lfloor \log_2 2021 \rfloor = 10$ and so its unary code is $00000000001$,
\item $x_r = x - 2^{\lfloor \log_2 2021 \rfloor } = 2021 - 2^{10} = 997$ which has the binary code $1111100101$,
\item So the {\bf $\gamma$-code} of $2021$ is $00000000001,1111100101$. 
\end{itemize}
\item 
\begin{itemize}
\item $x_d = \lfloor \log_2 2021 \rfloor = 10$, 
\item $x_{dd} = \lfloor \log_2(10+1) \rfloor = 3$, and so its unary code is $0001$,
\item $x_{dr} = (x_d+1) - 2^{x_{dd}} = (10+1) - 2^3 = 3$ which has the binary code $11$, 
\item $x_r = x - 2^{x_d} = 2021 - 2^{10} = 997$ with binary code $1111100101$,
\item So the {\bf $\delta$-code} of $2021$ is $0001,11,1111100101$. 
\end{itemize}
\item
\begin{itemize}
\item $000010100$ is encoded in $\gamma$-code and we can decode it to get $2^4+2^2=20$.
\item $001010101$ is encoded in $\delta$-code and we can decode it to get $2^4+(2^2+1)=21$. Calculations are as follows:

$x_{dd} = (001)_{\text{unary}} = 2 \Rightarrow 1 = (01)_2 = x_{dr} = (x_d+1) - 2^{x_{dd}} = x_d + 1 - 2^2 \Rightarrow$ $x_d = 4 \Rightarrow x = x_r + 2^{x_d} = (0101)_2 + 2^4 = 5 + 2^4 = 21$. 
\end{itemize}
\end{itemize}

























\end{document}
