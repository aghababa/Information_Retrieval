\documentclass[11pt]{article}

\usepackage{pgfplots}
\pgfplotsset{compat=1.12}
\newtheorem{define}{Definition}
%\newcommand{\Z}{{\mathbb{Z}}}
%\usepackage{psfig}
\usepackage{tikz}
\usetikzlibrary{arrows, automata}
\oddsidemargin=0.15in
\evensidemargin=0.15in
\topmargin=-.5in
\textheight=9in
\textwidth=6.25in
\usepackage{latexsym,bm}
\usepackage{amsmath}
\usepackage{amsfonts}%for math
\usepackage{graphicx}%for eps
\usepackage{url}


\newif\ifgrading

\gradingtrue
%\gradingfalse

\begin{document}
\input{preamble.tex}

\assignment{2}{}{u1255635}{Hasan Pourmahmoodaghababa}

\baselineskip = 5mm

%\documentclass[11pt]{article}

\usepackage{pgfplots}
\pgfplotsset{compat=1.12}
\newtheorem{define}{Definition}
%\newcommand{\Z}{{\mathbb{Z}}}
%\usepackage{psfig}
\usepackage{tikz}
\usetikzlibrary{arrows, automata}
\oddsidemargin=0.15in
\evensidemargin=0.15in
\topmargin=-.5in
\textheight=9in
\textwidth=6.25in
\usepackage{latexsym,bm}
\usepackage{amsmath}
\usepackage{amsfonts}%for math
\usepackage{graphicx}%for eps
\usepackage{url}


\newif\ifgrading

\gradingtrue
%\gradingfalse

\begin{document}
\input{preamble.tex}

\assignment{3}{}{u1255635}{Hasan Pourmahmoodaghababa}

\baselineskip = 5mm

%\documentclass[11pt]{article}

\usepackage{pgfplots}
\pgfplotsset{compat=1.12}
\newtheorem{define}{Definition}
%\newcommand{\Z}{{\mathbb{Z}}}
%\usepackage{psfig}
\usepackage{tikz}
\usetikzlibrary{arrows, automata}
\oddsidemargin=0.15in
\evensidemargin=0.15in
\topmargin=-.5in
\textheight=9in
\textwidth=6.25in
\usepackage{latexsym,bm}
\usepackage{amsmath}
\usepackage{amsfonts}%for math
\usepackage{graphicx}%for eps
\usepackage{url}


\newif\ifgrading

\gradingtrue
%\gradingfalse

\begin{document}
\input{preamble.tex}

\assignment{3}{}{u1255635}{Hasan Pourmahmoodaghababa}

\baselineskip = 5mm

%\documentclass[11pt]{article}

\usepackage{pgfplots}
\pgfplotsset{compat=1.12}
\newtheorem{define}{Definition}
%\newcommand{\Z}{{\mathbb{Z}}}
%\usepackage{psfig}
\usepackage{tikz}
\usetikzlibrary{arrows, automata}
\oddsidemargin=0.15in
\evensidemargin=0.15in
\topmargin=-.5in
\textheight=9in
\textwidth=6.25in
\usepackage{latexsym,bm}
\usepackage{amsmath}
\usepackage{amsfonts}%for math
\usepackage{graphicx}%for eps
\usepackage{url}


\newif\ifgrading

\gradingtrue
%\gradingfalse

\begin{document}
\input{preamble.tex}

\assignment{3}{}{u1255635}{Hasan Pourmahmoodaghababa}

\baselineskip = 5mm

%\input{answers}

\section*{Ranking Features}%

\begin{itemize}
\item Query-document matching features: sum of tf$*$idf, mean of term frequency, BM25.
\item Document quality features: QualityScore, QualityScore2.
\item User behavior features: url dwell time, url click count, Query-url click count.
\end{itemize}
About which type of features are expected to be most useful in practice, I think it depends on many things like the query and the intend of user. 
\begin{itemize}
\item There are cases that query-document matching features are expected to be most useful in practice (like when we are searching for the query "Facebook"). 
\item There are cases that document quality features are very important (for example, when one searches for an app to install on his/her machine. In this case the quality of a web becomes essential).
\item In other situations like searching for an ambiguous query (probably like "Java") with more than one meaning, in addition to query-document matching features, user behavior features also can be effective to help semantic matching. 
\end{itemize}

%It seems that Query-document matching features are expected to be most useful in practice as more than $90\%$ of features are of this type which shows their importance. 



\section*{Ranking Models}%

\begin{itemize}
\item SVMrank: Pairwise
\item RankNet: Pairwise
\item MART: Pointwise
\item LamdbaMART: Listwise
\end{itemize}
For nDCG@5 I got the following values:
\begin{itemize}
\item SVMrank: 0.2975 (using linear normalization and sorted data by qid. I wrote a code in python to calculate nDCG@5 from predictions generated by SVMrank for test data.)
\item RankNet: 0.2893 (using zscore normalization, 1 layer, 5 nodes and learning rate of 0.00001)
\item MART: 0.4294 (using linear normalization)
\item LamdbaMART: 0.4311 (using linear normalization)
\end{itemize}
LamdbaMART outperformed other models here. I think the reason is that it is a listwise method and thus is based on ranking metrics (like MAP, nDCG) on list level. Therefore, to maximize ranking performance, this model tries to optimize the ranking list directly. However, in pointwise and pairwise models the loss functions we try to optimize are not the ranking metrics. Therefore, it is expected to get a better performance with listwise models like LamdbaMART than pointwise and pairwise models like SVMrank, RankNet and MART. 












\end{document}


\section*{Ranking Features}%

\begin{itemize}
\item Query-document matching features: sum of tf$*$idf, mean of term frequency, BM25.
\item Document quality features: QualityScore, QualityScore2.
\item User behavior features: url dwell time, url click count, Query-url click count.
\end{itemize}
About which type of features are expected to be most useful in practice, I think it depends on many things like the query and the intend of user. 
\begin{itemize}
\item There are cases that query-document matching features are expected to be most useful in practice (like when we are searching for the query "Facebook"). 
\item There are cases that document quality features are very important (for example, when one searches for an app to install on his/her machine. In this case the quality of a web becomes essential).
\item In other situations like searching for an ambiguous query (probably like "Java") with more than one meaning, in addition to query-document matching features, user behavior features also can be effective to help semantic matching. 
\end{itemize}

%It seems that Query-document matching features are expected to be most useful in practice as more than $90\%$ of features are of this type which shows their importance. 



\section*{Ranking Models}%

\begin{itemize}
\item SVMrank: Pairwise
\item RankNet: Pairwise
\item MART: Pointwise
\item LamdbaMART: Listwise
\end{itemize}
For nDCG@5 I got the following values:
\begin{itemize}
\item SVMrank: 0.2975 (using linear normalization and sorted data by qid. I wrote a code in python to calculate nDCG@5 from predictions generated by SVMrank for test data.)
\item RankNet: 0.2893 (using zscore normalization, 1 layer, 5 nodes and learning rate of 0.00001)
\item MART: 0.4294 (using linear normalization)
\item LamdbaMART: 0.4311 (using linear normalization)
\end{itemize}
LamdbaMART outperformed other models here. I think the reason is that it is a listwise method and thus is based on ranking metrics (like MAP, nDCG) on list level. Therefore, to maximize ranking performance, this model tries to optimize the ranking list directly. However, in pointwise and pairwise models the loss functions we try to optimize are not the ranking metrics. Therefore, it is expected to get a better performance with listwise models like LamdbaMART than pointwise and pairwise models like SVMrank, RankNet and MART. 












\end{document}


\section*{Ranking Features}%

\begin{itemize}
\item Query-document matching features: sum of tf$*$idf, mean of term frequency, BM25.
\item Document quality features: QualityScore, QualityScore2.
\item User behavior features: url dwell time, url click count, Query-url click count.
\end{itemize}
About which type of features are expected to be most useful in practice, I think it depends on many things like the query and the intend of user. 
\begin{itemize}
\item There are cases that query-document matching features are expected to be most useful in practice (like when we are searching for the query "Facebook"). 
\item There are cases that document quality features are very important (for example, when one searches for an app to install on his/her machine. In this case the quality of a web becomes essential).
\item In other situations like searching for an ambiguous query (probably like "Java") with more than one meaning, in addition to query-document matching features, user behavior features also can be effective to help semantic matching. 
\end{itemize}

%It seems that Query-document matching features are expected to be most useful in practice as more than $90\%$ of features are of this type which shows their importance. 



\section*{Ranking Models}%

\begin{itemize}
\item SVMrank: Pairwise
\item RankNet: Pairwise
\item MART: Pointwise
\item LamdbaMART: Listwise
\end{itemize}
For nDCG@5 I got the following values:
\begin{itemize}
\item SVMrank: 0.2975 (using linear normalization and sorted data by qid. I wrote a code in python to calculate nDCG@5 from predictions generated by SVMrank for test data.)
\item RankNet: 0.2893 (using zscore normalization, 1 layer, 5 nodes and learning rate of 0.00001)
\item MART: 0.4294 (using linear normalization)
\item LamdbaMART: 0.4311 (using linear normalization)
\end{itemize}
LamdbaMART outperformed other models here. I think the reason is that it is a listwise method and thus is based on ranking metrics (like MAP, nDCG) on list level. Therefore, to maximize ranking performance, this model tries to optimize the ranking list directly. However, in pointwise and pairwise models the loss functions we try to optimize are not the ranking metrics. Therefore, it is expected to get a better performance with listwise models like LamdbaMART than pointwise and pairwise models like SVMrank, RankNet and MART. 












\end{document}


\section*{Task 1.}%

The values I got are: 

\begin{itemize}
\item with dirichelet smoothing and krovetz stemming: 0.250
\item with dirichelet smoothing but without krovetz stemming:0.225
\item with jm smoothing and krovetz stemming: 0.228
\item with jm smoothing but without krovetz stemming:0.203
\item without smoothing with krovetz stemming: 0.159
\item without smoothing but without krovetz stemming: 0.131 
\end{itemize}

I learnt smoothing improves MAP by a significant amount in general. Krovetz stemming also improves MAP in a fair amount but not as significant as smoothing. However, different smoothing techniques have different degrees of improvement. For example, in this task, we observed that dirichlet smoothing improves MAP more than jm smoothing. One possible reason is the coefficient $\lambda$ in jm smoothing is fixed for all documents in the corpus but the dirichlet smoothing considers $\lambda$ according to the document size for each document (i.e. $\lambda$ differs for each document) and so it is more effective. 

\section*{Task 2.}
 
\begin{itemize}
\item[(i)] MAP value for RM1 with 25 top ranked docs and 100 top feedback terms: {\bf 0.193}

\item[(ii)] MAP value for RM3 with 25 top ranked docs, 100 top feedback terms and 0.3 for the weight of the original query: {\bf 0.293}

\item[(iii)] MAP value for RM3 with 10 top ranked docs, 20 top feedback terms and 0.3 for the weight of the original query: {\bf 0.271}

\item[(iv)] MAP value for RM3 with 25 top ranked docs, 100 top feedback terms and 0.7 for the weight of the original query: {\bf 0.268}
\end{itemize}

From these results (specially from (i), (ii) and (iii)) first I learnt that RM3 works better than RM1 as the MAP is significantly higher. Comparing (ii) and (iii) shows that using more top ranked docs and more top feedbacks provides a higher performance (i.e. improves overall precision). Lastly, looking at (ii) and (iv) we observe that the lower weight for the weight of the original query performs better than bigger weights. This is also shown in comparing (iii) and (iv); even though smaller number of top ranked docs and top feedbacks are used in (iii) than (iv), it performed a bit better as the weight for the weight of the original query in (iii) was smaller than the one for (iv).

\section*{Question 3.}%


The idea of the Mixture model is to linearly combine the probability of the term generated from the language model of the feedback document ($F$) and the probability of the term generated from the collection ($C$), that is, 
$$p(t) = (1 - \lambda) p(t| \theta_F) + \lambda p(t|\theta_C).$$
The EM algorithm has two steps, namely E-step and M-step, which will be updated iteratively. 

The E-step is calculating some latent variables based on data, which here for a latent random variable $Z_t$ is 
$$p(z_t=1) = \frac{(1 - \lambda) p(t| \theta_F)}{(1 - \lambda) p(t| \theta_F) + \lambda p(t|\theta_C)}.$$
The M-step is maximizing the expectation of the probabilities that we have got from E-step. It turns out that after maximizing the expectation we get the following formula to calculate the probability of the term generated from the language model of the feedback document:
$$p(t| \theta_F) = \frac{count(t, F) p(z_t=1)}{\sum_{t' \in V} count(t', F) p(Z_{t'}=1)}.$$

















\end{document}
